\documentclass{article}

\usepackage{fancyhdr} % Required for custom headers
\usepackage{lastpage} % Required to determine the last page for the footer
\usepackage{extramarks} % Required for headers and footers
\usepackage{graphicx} % Required to insert images
\usepackage{tikz}
\usepackage{alltt}
\usepackage{url}

\usetikzlibrary{shapes.multipart,positioning}

% Margins
\topmargin=-0.45in
\evensidemargin=0in
\oddsidemargin=0in
\textwidth=6.5in
\textheight=9.0in
\headsep=0.25in 

\linespread{1.1} % Line spacing

% Set up the header and footer
\pagestyle{fancy}
\chead{\hmwkClass : \hmwkTitle} % Top center header
\rhead{\firstxmark} % Top right header
\lfoot{\lastxmark} % Bottom left footer
\cfoot{} % Bottom center footer
\rfoot{Page\ \thepage\ of\ \pageref{LastPage}} % Bottom right footer
\renewcommand\headrulewidth{0.4pt} % Size of the header rule
\renewcommand\footrulewidth{0.4pt} % Size of the footer rule

\newcommand{\hmwkTitle}{Homework\ \#2} % Assignment title
\newcommand{\hmwkClass}{Advanced Networking,\ Fall 2016} % Course/class

\tikzset{
    font=\sffamily,
    BLOCK/.style={
        draw,
        align=center,
        text height=0.4cm,
        draw=gray!50,
        fill=gray!20,
        rectangle split, 
        rectangle split horizontal,
        rectangle split parts=#1, 
    }
}

\title{foo}

%----------------------------------------------------------------------------------------

\begin{document}

%\maketitle


%----------------------------------------------------------------------------------------
%	PROBLEM 1
%----------------------------------------------------------------------------------------

% To have just one problem per page, simply put a \clearpage after each problem

\section*{Assignment}

In this assignment, you will use P4 to implement a custom protocol
that tracks the path UDP packets travel through a network. Your custom
traceroute header will look like this:

\begin{alltt}
 num\_valid (4 bytes) | port\_1 (1 byte) | port\_2 (1 byte) | ... | port\_n (1 byte) | payload
\end{alltt}

\noindent
When a switch receives an EasyTrace packet, it should append the outgoing port and increment num\_valid by 1. When the recipient receives the packet, it should print the sequence of ports that the packet traversed.

The TA has implemented much of the needed infrastructure for you, and all of the code is available:

\begin{alltt}
https://github.com/usi-systems/easytrace
\end{alltt}

\begin{enumerate}
\item Read in a topology description of a network.

\end{enumerate}

\section*{What to hand in}
\begin{itemize}
\item todo
\end{itemize}


\end{document}
